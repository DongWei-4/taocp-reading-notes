
\documentclass[openany,UTF8]{ctexbook}
\usepackage{titlesec}
\usepackage{titletoc}
\usepackage{amsmath}
\usepackage{amsthm}

\renewcommand{\thesection}{\arabic{chapter}.\arabic{section}}
\titleformat{\section}[block]{\normalfont\Large\bfseries}{\thesection}{1em}{}


\title{计算机程序设计艺术\\卷1:基本算法\\读书笔记}
\author{DongWei}
\date{}

\begin{document}
\pagenumbering{roman}
\maketitle

\tableofcontents

\newpage

\pagenumbering{arabic} 
\chapter{基本概念}
\section{算法}

{欧几里得算法证明}
欧几里得算法的证明过程如下

\begin{proof}[欧几里得算法]
  已知,在每次迭代过程中
  \begin{equation}
    m=qn+r
  \end{equation}

  在$r\neq0$的情况下,假设$m$和$n$有公约数$p$,带入式(1.1)得
  \begin{equation}
    \begin{aligned}
      &ap=bqp+r \\
      &(a-bq)p=r
    \end{aligned}
  \end{equation}
  
  因此$p$也是$r$的约数,所以任何$(m,n)$的公约数,都是$(n,r)$的公约数
  
  类似方法可证,任何$(n,r)$的公约数,都是$(m,n)$的公约数
  
  因此$(m,n)$和$(n,r)$的公约数集完全相同,最大公约数也相同
\end{proof}

\section{数学准备}
这是第一章的第二个小节的内容。


\end{document}


