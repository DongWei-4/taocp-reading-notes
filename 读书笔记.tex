
\documentclass[openany,UTF8]{ctexbook}
\usepackage{titlesec}
\usepackage{titletoc}
\usepackage{amsmath}
\usepackage{amsthm}
\usepackage{enumitem}
\usepackage{geometry}

\geometry{top=1in, bottom=1in, left=0.75in, right=0.75in}

\renewcommand{\thesection}{\arabic{chapter}.\arabic{section}}
\titleformat{\section}[block]{\normalfont\Large\bfseries}{\thesection}{1em}{}


\title{计算机程序设计艺术\\卷1:基本算法\\读书笔记}
\author{DongWei}
\date{}

\begin{document}
\pagenumbering{roman}
\maketitle

\tableofcontents

\newpage

\pagenumbering{arabic} 
\chapter{基本概念}
\section{算法}

\paragraph{欧几里得算法证明}
欧几里得算法的证明过程如下
\begin{proof}[欧几里得算法证明]
  已知,在每次迭代过程中
  \begin{equation}
    m=qn+r
  \end{equation}

  在$r\neq0$的情况下,假设$m$和$n$有公约数$p$,带入式(1.1)得
  \begin{equation}
    \begin{aligned}
      &ap=bqp+r \\
      &r=(a-bq)p
    \end{aligned}
  \end{equation}
  
  因此$p$也是$r$的约数,所以任何$(m,n)$的公约数,都是$(n,r)$的公约数
  
  类似方法可证,任何$(n,r)$的公约数,都是$(m,n)$的公约数
  
  因此$(m,n)$和$(n,r)$的公约数集完全相同,最大公约数也相同
\end{proof}

\paragraph{计算方法}
书中所述计算方法定义,理解要点如下
\begin{enumerate}
\item $f$在$\Omega$上点点不动:$\Omega$为输出域,$f(q)=q$意思为获得输出后,终止计算
\item 计算序列:就是计算步骤,步骤间可以跳转,计算序列在$k$步内终止就是一共有$k$个计算步骤
\item $p$表示含义:$r>0$才会跳转到计算步骤3,所以用了新的正整数$p$代替$r$
\item 字符串相关计算方法:结合习题8理解,N即表示共有N个计算步骤,$\theta_j$、$\phi_j$、$a_j$、$b_j$均表示常量,每个计算步骤的常量不同
\end{enumerate}
  
\paragraph{习题}
习题解答
\begin{enumerate}
\item $t <- a <- b <- c <- d <- t $
\item $r=m \bmod n$,因此$r$必然小于$n$
\item 不使用$r$缓存取模结果,轮流用$m$或$n$去缓存结果,$m=m \bmod n$与$n=n \bmod m$交替进行
\item 57
\item 除了输入,啥也没有
\item 5是质数
  对无穷大的数来说,执行一次$\bmod 5$得到的结果为0~4,且概率均等
  再计算对0~4,求与5的公约数,执行次数为1、2、3、4、3,平均为2.6
\item $m$已知,对于无限大取值范围的$n$来说,$n>m$的概率为无穷大,因此考虑平均执行次数,只需要考虑$n>m$的情况
  对于$n>m$,执行一次计算后,实际上交换了$n$和$m$,因此变成了$n$已知,$m$无限的求公约数计算,故$U_m=T_m+1$
\item 用更相减损法,计算最大公约数
  \begin{enumerate}[label=\arabic*)] 
  \item 计算$\left| m-n \right|$,若$\sigma$中有ab,则删掉ab,跳转到步骤1),若没有ab,跳转到步骤2)
  \item 计算$min(m,n)$,左侧增加一个c,用于记录减法的次数,当减法执行完成,c的个数就等于较小值,跳到步骤0)
  \item $n<-\left| m-n \right|$,若存在a,a替换为b,跳到步骤2),否则跳到步骤3)
  \item $m<- min(m-n)$,c转化为a,若存在c,c替换为a,跳到步骤3),否则跳到步骤4)
  \item 判断$\left| m-n \right|$是否为0,检查是否有b,有b跳到步骤0),否则结束
  \end{enumerate}
  因此,若用矩阵表示每个计算步骤的常量如下\\
  \begin{center}
    \begin{tabular}{c|c|c|c|c}
      \hline
      $j$&$\theta_j$&$\phi_j$&$a_j$&$b_j$\\
      \hline
      0&ab&空&1&2\\
      1&空&c&0&不可能\\
      2&a&b&2&3\\
      3&c&a&3&4\\
      a4&b&b&0&5\\
      \hline
    \end{tabular}
  \end{center}
\item 这也要证?还是看完集合论再说吧
\end{enumerate}

\section{数学准备}

\end{document}


